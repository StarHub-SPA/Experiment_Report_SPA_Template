\documentclass{spaexp}

%下面这几个是报告的信息
\major{物理学}
\name{StarHub\footnote{\url{StarHub@qun.mail.163.com}}}
\grade{2017级}
\stuid{18353xxx}
\exptitle{SH1 StarHub实验}

% \setCJKmainfont{\songti}
\begin{document}
\maketitle %生成报告的开头表格。
\section{背景介绍}
    这里是一个SPA实验模板(非官方)的简单使用示例。
    本例子仅仅介绍一些比较常见的用法,供入坑的朋友参考。这个示例的用法推荐是,边看这个pdf,然后一边看 \TeX 源码,对比着来看,
    然后照葫芦画瓢。这需要一丢丢的 \LaTeX 的基本知识,如果发现自己完全一点也看不懂 \TeX 源码到底在干啥,那么推荐是先迅速看一点
    写 \LaTeX 的基础知识,推荐《一份不太简短的\LaTeX2e 介绍》。把1.2节看完再加上一些推理能力,相信就能看懂代码了。其他内容
    可以当作工具书查询,对于使用这个模板并不是刚需。
    \subsection{问题}
        噢!模版的使用需要用xeLaTex 引擎进行编译。如果你不知道这个是什么,大部分时候把你编译的工具上的pdflatex 换成xeLaTex 
        就可以了。至于引擎不引擎的,想知道的看上面那本书的第一章就会有个概念。\par
        
        当编译出现问题的时候,首先检查用没用xeLaTex ,其次再找身边的朋友帮忙debug。最后再联系StarHub,我们应该会比较乐意解答你的问题。
        当然,希望大家互相之间可以学习交流,毕竟跟StarHub联系交流周期长,不利于赶实验报告DDL。
    \subsection{反馈}
        如果你在使用了这个模板之后,不管有好的或者不好的感受,都希望能给予我们一些反馈,以便我们开心一下或者更新一下。当然,你要是觉得哪里用的不方便,
        或者说觉得哪些地方很难看,想改,都可以给我们反馈,我们都会细心聆听。如果是SPA实验课老师认为模板有不符合要求的地方,请务必联系我们。

\section{实验原理}
    这里介绍一些基础功能。交叉引用加上label,然后用autoref 。\autoref{tab:Xtab}

    \subsection{分点}
        \begin{enumerate}
            \item 这是带编号的分点环境。
            \item 还可以俄罗斯套娃
            \item \begin{enumerate}
                \item 一套
            \end{enumerate}
        \end{enumerate}

        \begin{itemize}
            \item 这个不带编号
            \item 也可以套娃。
        \end{itemize}

        \begin{description}
            \item[名词] 这个有点像名词解释
            \item[叫做] 叫做description环境。
        \end{description}

    \subsection{表格}
        \begin{table}[h]
            \centering
            \caption{在这里加标题}
            \begin{tabular}{c|c}
                \toprule
                这是个基础普通的表格 & 一些语法可能要查书才会噢。\\
                \midrule
                表格使用在\LaTeX 里面较为繁琐 & 大家慢慢学习 \\
                \bottomrule
            \end{tabular}
        \end{table}
            
        \begin{table}[h]
            \centering
            \caption{这是一个可以控制长度的均分表格}\label{tab:Xtab}
            \begin{tabularx}{0.85\textwidth}{|X|X|X|}
                \toprule
                这是一个叫做tabularx 宏包 & 提供的功能挺方便 & 用法并不难 \\ \hline
                第一个参数为长度 & 通常为都用页面长度的倍数来表示 & 比如: \verb|0.75\textwidth| \\ \hline
                更多资料请看 & tabularx & 帮助文档 \\ \hline
            \end{tabularx}
        \end{table}

        \begin{longtable}{p{0.7\textwidth}||p{0.3\textwidth}}
            \caption{这是一个可以跨页的表格}\\
            \toprule
            如果数据记录或者什么的表格太长 & 它可以自动跨页排版 \\ \hline
            比如这样 & 请看 \\ \hline
            & \\ \hline
            & \\ \hline
            & \\ \hline
            & \\ \hline
            & \\ \hline
            长啊长 & \\ \hline
            & \\ \hline
            & \\ \hline
            & \\ \hline
            有时候觉得徒手打表格有点累&  \\ \hline
            尤其是还要把实验数据打进去& \\ \hline
            如果想偷懒的话& \\ \hline
            可以了解一下\url{https://github.com/Benature/AutoLaTeX}这个仓库& \\ \hline
            & \\ \hline
            & \\ \hline
            如你所见 & 他十分之长 \\ 
            \bottomrule 
        \end{longtable}
        这里我们插入一个换页符
        \clearpage
    \subsection{插入图片}
        \begin{figure}[h]
            \ct
            \caption{插入图片的功能}\label{fig:handsomeBOY}
            \includegraphics[width = 0.5\textwidth]{img/data1.png}
        \end{figure}

    \subsection{数学公式}
        作为\LaTeX 的招牌项目我就不仔细展开了。网上很多教程。
        就粗略总结一下:
        \begin{description}
            \item[equation] 普通方程环境
            \item[align] 递等式,主要是对齐比较好用
            \item[gather] 几条方程堆在一起,不对齐
        \end{description}
    
\section{实验方案}
    实验方案这部分,考虑到需要老师对实验报告的要求,整合了一些常用功能。主要是排版审美功力
    不足,自己都觉得有点丑,大家有更好的设计欢迎提供,说不定就给你更新用上了。(只要有个想法
    画个图就可以,具体实现我们会尽力的)

    \subsection{xx实验方案}
        方案思路…………

        \recordhead
        \begin{step}
            \movehead
            \move{打开电脑}
            \move{关闭电脑}
            \move{背上书包,立马走人}
        \end{step}

\section{分析与讨论}
    \subsection{思考题}
        模板支持三种环境。
        \begin{que}
            这个是问题环境que
        \end{que}
        \vspace{1cm}
        上下是两个独立的环境
        \vspace{1cm}
        \begin{ans}
            这个是回答环境ans
        \end{ans}

        以及一个qna环境
        \begin{qna}{这里放问题}
            这里是回答。
        \end{qna}
\end{document}